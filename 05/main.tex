\documentclass[fleqn,onecolumn,a4paper,12pt,titlepage]{article}
\usepackage[showcrop]{geometry}
\geometry{textheight=25.0cm,textwidth=17.0cm,headheight=3cm,headsep=0.5em}%hmargin=4.2cm
\usepackage{graphicx}%[dvips,pdftex]
\usepackage{amssymb}
\usepackage{amsmath}%centertags
\usepackage[T1]{fontenc}
\usepackage[polish]{babel}
\usepackage{float}
\usepackage[final]{listingsutf8}
\usepackage{fancyhdr}
\usepackage{tikz}
\usepackage{caption}
\usepackage{hyperref}
\usepackage{datetime2}
\usepackage{enumitem}
\usepackage{blindtext}

\newcommand{\kod}[1]{\texttt{#1}}

\pagestyle{fancy}
\fancyfoot[L,R]{\tiny{\texttt{\symbol{64}}  built: \today{} at \DTMcurrenttime}} 

\begin{document}

\begin{titlepage}
    \includegraphics[width=0.25\textwidth]{logo_kia.png}\par\vspace{3cm}
    \centering
    {\LARGE \textsc{Narzędzia dla programistów} \par}
    \vspace{2cm}
    {\Large \textsc{Laboratorium 5} \par} % NALEŻY UZUPEŁNIĆ
    \vspace{2cm}
    {\textsc{Pakiety LATEX i własne makra} \par} % NALEŻY UZUPEŁNIĆ
    \vfill
    Mikołaj Hus $-$ 179503 $-$ L06 \par % NALEŻY UZUPEŁNIĆ
    \vspace{2cm}
    {\large {\today} \par}
\end{titlepage}

\section*{Zadanie 1}
Przygotowanie do pracy.

\section*{Zadanie 2}
Zbudowanie projektu LATEX.
\begin{figure}[H]%
    \centering\includegraphics[width=0.9\textwidth]{02.png}
\end{figure}

\section*{Zadanie 3}
Dodanie obsługi pakietu \textit{caption}.
\begin{figure}[H]%
    \centering\includegraphics[width=0.9\textwidth]{03.png}
\end{figure}

\section*{Zadanie 4}
Modyfikacja style podpisów dla tabel.
\begin{figure}[H]%
    \centering\includegraphics[width=0.9\textwidth]{04.png}
\end{figure}

\section*{Zadanie 5}
Dodanie kropek po numerach punktów i podpunktów.
\begin{figure}[H]%
    \centering\includegraphics[width=0.9\textwidth]{05.png}
\end{figure}

\section*{Zadanie 6}
Zmiana stylu wyswietlania numerów równań.
\begin{figure}[H]%
    \centering\includegraphics[width=0.9\textwidth]{06.png}
\end{figure}

\section*{Zadanie 7}
Utworzenie nowego polecenia.
\begin{figure}[H]%
    \centering\includegraphics[width=0.9\textwidth]{07.png}
\end{figure}

\section*{Zadanie 8}
Zmiana stylu wypunktowania.
\begin{figure}[H]%
    \centering\includegraphics[width=0.9\textwidth]{08.png}
\end{figure}

\section*{Zadanie 9}
Dodanie nowej listy wypunktowanej.
\begin{figure}[H]%
    \centering\includegraphics[width=0.9\textwidth]{09.png}
\end{figure}

\section*{Zadanie 10}
Dodanie układu równań.
\begin{figure}[H]%
    \centering\includegraphics[width=0.9\textwidth]{10.png}
\end{figure}

\section*{Zadanie 11}
Dodanie odwołania do wzrou.
\begin{figure}[H]%
    \centering\includegraphics[width=0.9\textwidth]{11.png}
\end{figure}

\section*{Zadanie 12}
Zdefiniowanie własnego koloru.
\begin{figure}[H]%
    \centering\includegraphics[width=0.9\textwidth]{12.png}
\end{figure}

\section*{Zadanie 13}
Użycie nowego koloru.
\begin{figure}[H]%
    \centering\includegraphics[width=0.9\textwidth]{13.png}
\end{figure}

\section*{Zadanie 14}
Dołaczenie kodu źródłowego z innego pliku.
\begin{figure}[H]%
    \centering\includegraphics[width=0.9\textwidth]{14.png}
\end{figure}

\section*{Zadanie 15}
Zmiana języka oraz kodowania.
\begin{figure}[H]%
    \centering\includegraphics[width=0.9\textwidth]{15.png}
\end{figure}

\section*{Zadanie 16}
Zmiana domyślnego wyświetlania listingów.
\begin{figure}[H]%
    \centering\includegraphics[width=0.9\textwidth]{16.png}
\end{figure}

\section*{Zadanie 17}
Zmiana danych pliku PDF.
\begin{figure}[H]%
    \centering\includegraphics[width=0.9\textwidth]{17.png}
\end{figure}

\section*{Zadanie 18}
Dodanie załącznika do pliku.
\begin{figure}[H]%
    \centering\includegraphics[width=0.9\textwidth]{18.png}
\end{figure}

\section*{Zadanie 19}
Dołączenie bibliografi.
\begin{figure}[H]%
    \centering\includegraphics[width=0.9\textwidth]{19.png}
\end{figure}

\section*{Zadanie 20}
Importowanie bazy literatury.
\begin{figure}[H]%
    \centering\includegraphics[width=0.9\textwidth]{20.png}
\end{figure}

\section*{Zadanie 21}
Skompilowanie dokumentu i sprawdzenie bibliografi.
\begin{figure}[H]%
    \centering\includegraphics[width=0.9\textwidth]{21.png}
\end{figure}

\section*{Zadanie 22}
Dodanie nowej pozycji do bibliografi.
\begin{figure}[H]%
    \centering\includegraphics[width=0.9\textwidth]{22.png}
\end{figure}

\section*{Zadanie 23}
Użycie cytowania.
\begin{figure}[H]%
    \centering\includegraphics[width=0.9\textwidth]{23.png}
\end{figure}

\section*{Zadanie 24}
Zmiana formatu cytowania.
\begin{figure}[H]%
    \centering\includegraphics[width=0.9\textwidth]{24.png}
\end{figure}

\section*{Zadanie 25}
Modyfikacja obwodu elektronicznego.
\begin{figure}[H]%
    \centering\includegraphics[width=0.9\textwidth]{25.png}
\end{figure}

\end{document}

