\documentclass[fleqn,onecolumn,a4paper,12pt,titlepage]{article}
\usepackage[showcrop]{geometry}
\geometry{textheight=25.0cm,textwidth=17.0cm,headheight=3cm,headsep=0.5em}%hmargin=4.2cm
\usepackage{graphicx}%[dvips,pdftex]
\usepackage{amssymb}
\usepackage{amsmath}%centertags
\usepackage[T1]{fontenc}
\usepackage[polish]{babel}
\usepackage{float}
\usepackage[final]{listingsutf8}
\usepackage{fancyhdr}
\usepackage{tikz}
\usepackage{caption}
\usepackage{hyperref}
\usepackage{datetime2}
\usepackage{enumitem}
\usepackage{blindtext}
\usepackage{listings}
\usepackage{color}

\definecolor{dkgreen}{rgb}{0,0.6,0}
\definecolor{gray}{rgb}{0.5,0.5,0.5}
\definecolor{mauve}{rgb}{0.58,0,0.82}

\lstset{frame=tb,
  language=Bash,
  aboveskip=3mm,
  belowskip=3mm,
  showstringspaces=false,
  columns=flexible,
  basicstyle={\small\ttfamily},
  numbers=none,
  numberstyle=\tiny\color{gray},
  keywordstyle=\color{blue},
  commentstyle=\color{dkgreen},
  stringstyle=\color{mauve},
  breaklines=true,
  breakatwhitespace=true,
  tabsize=3
}

\pagestyle{fancy}
\fancyfoot[L,R]{\tiny{\texttt{\symbol{64}}  built: \today{} at \DTMcurrenttime}} 

\begin{document}

\begin{titlepage}
    \includegraphics[width=0.25\textwidth]{kia.png}\par\vspace{3cm}
    \centering
    {\LARGE \textsc{Narzędzia dla programistów} \par}
    \vspace{2cm}
    {\Large \textsc{Laboratorium 2} \par} % NALEŻY UZUPEŁNIĆ
    \vspace{2cm}
    {\textsc{Polecenia powłoki Bash} \par} % NALEŻY UZUPEŁNIĆ
    \vfill
    Mikołaj Hus $-$ 179503 $-$ L06 \par % NALEŻY UZUPEŁNIĆ
    \vspace{2cm}
    {\large {\today} \par}
\end{titlepage}

\section*{Zadanie 1}
Zapoznanie się z terminalem.
\begin{figure}[H]%
    \centering\includegraphics[width=0.9\textwidth]{01.png}
\end{figure}

\section*{Zadanie 2}
Zapoznanie się podstawowymi poleceniami.
\begin{figure}[H]%
    \centering\includegraphics[width=0.9\textwidth]{021.png}
\end{figure}
\begin{figure}[H]%
    \centering\includegraphics[width=0.9\textwidth]{022.png}
\end{figure}
\begin{figure}[H]%
    \centering\includegraphics[width=0.9\textwidth]{023.png}
\end{figure}
\begin{figure}[H]%
    \centering\includegraphics[width=0.9\textwidth]{024.png}
\end{figure}
\begin{figure}[H]%
    \centering\includegraphics[width=0.9\textwidth]{025.png}
\end{figure}
\begin{figure}[H]%
    \centering\includegraphics[width=0.9\textwidth]{026.png}
\end{figure}
\begin{figure}[H]%
    \centering\includegraphics[width=0.9\textwidth]{027.png}
\end{figure}
\begin{figure}[H]%
    \centering\includegraphics[width=0.9\textwidth]{028.png}
\end{figure}

\section*{Zadanie 3}
Utworzenie struktury folderów.
\begin{figure}[H]%
    \centering\includegraphics[width=0.9\textwidth]{03.png}
\end{figure}

\section*{Zadanie 4}
Przekierowanie wejścia do pliku za pomocą polecenia \textit{cat}.
\begin{figure}[H]%
    \centering\includegraphics[width=0.9\textwidth]{04.png}
\end{figure}

\section*{Zadanie 5}
Wyświetlenie zawartości pliku.
\begin{figure}[H]%
    \centering\includegraphics[width=0.9\textwidth]{05.png}
\end{figure}

\section*{Zadanie 6}
Przekierowanie polecenia \textit{date} do pliku.
\begin{figure}[H]%
    \centering\includegraphics[width=0.9\textwidth]{06.png}
\end{figure}

\section*{Zadanie 7}
Sprawdzenie pobranej godziny.
\begin{figure}[H]%
    \centering\includegraphics[width=0.9\textwidth]{07.png}
\end{figure}

\section*{Zadanie 8}
Wyświetlenie zawartości katalogów.
\begin{figure}[H]%
    \centering\includegraphics[width=0.9\textwidth]{08.png}
\end{figure}

\section*{Zadanie 9}
Połączenie dwóch plików tekstowych.
\begin{figure}[H]%
    \centering\includegraphics[width=0.9\textwidth]{09.png}
\end{figure}

\section*{Zadanie 10}
Utworzenie plików ze znakami specjalnymi w nazwie.
\begin{figure}[H]%
    \centering\includegraphics[width=0.9\textwidth]{10.png}
\end{figure}

\section*{Zadanie 11}
Przeniesienie plików ze znakami specjalnymi w nazwie.
\begin{figure}[H]%
    \centering\includegraphics[width=0.9\textwidth]{11.png}
\end{figure}

\section*{Zadanie 12}
Zmiana nazwy pliku ze znakami specjalnymi w nazwie.
\begin{figure}[H]%
    \centering\includegraphics[width=0.9\textwidth]{12.png}
\end{figure}

\section*{Zadanie 13}
Usunięcie plików ze znakami specjalnymi w nazwie.
\begin{figure}[H]%
    \centering\includegraphics[width=0.9\textwidth]{13.png}
\end{figure}

\section*{Zadanie 14}
Skopiowanie zawartości katalogu.
\begin{figure}[H]%
    \centering\includegraphics[width=0.9\textwidth]{14.png}
\end{figure}

\section*{Zadanie 15}
Wyświetlenie zmiennej systemowej \textit{PATH}.
\begin{figure}[H]%
    \centering\includegraphics[width=0.9\textwidth]{15.png}
\end{figure}

\section*{Zadanie 16}
Wyświetlenie zmiennej systemowej \textit{PATH} z podziałem na linie.
\begin{figure}[H]%
    \centering\includegraphics[width=0.9\textwidth]{16.png}
\end{figure}

\section*{Zadanie 17}
Zapisanie obecnego położenia do pliku.
\begin{figure}[H]%
    \centering\includegraphics[width=0.9\textwidth]{17.png}
\end{figure}

\section*{Zadanie 18}
Przekierowanie zawartości katalogu domowego do pliku.
\begin{figure}[H]%
    \centering\includegraphics[width=0.9\textwidth]{18.png}
\end{figure}

\section*{Zadanie 19}
Sprawdzenie zawartości folderu z użyciem rekurencji i filtrowania.
\begin{figure}[H]%
    \centering\includegraphics[width=0.9\textwidth]{19.png}
\end{figure}

\section*{Zadanie 20}
Wyświetlenie zawartości pliku z użyciem polecenie \textit{less}.
Na liście nie pojawia się plik rozpoczęty kropką ponieważ w taki sposób oznaczane są pliku ukryte.
\begin{figure}[H]%
    \centering\includegraphics[width=0.9\textwidth]{20.png}
\end{figure}

\section*{Zadanie 21}
Usunięcie całej zawartości folderu.
\begin{figure}[H]%
    \centering\includegraphics[width=0.9\textwidth]{21.png}
\end{figure}

\section*{Zadanie 22}
Powrót do zapamiętanego katalogu.
\begin{figure}[H]%
    \centering\includegraphics[width=0.9\textwidth]{22.png}
\end{figure}

\section*{Zadanie 23}
Wyświetlenie zmiennych systemowych.
\begin{figure}[H]%
    \centering\includegraphics[width=0.9\textwidth]{23.png}
\end{figure}

\section*{Zadanie 24}
Napisanie skryptu sumującego 2 liczby.
\begin{figure}[H]%
    \centering\includegraphics[width=0.9\textwidth]{241.png}
\end{figure}
\begin{figure}[H]%
    \centering\includegraphics[width=0.9\textwidth]{242.png}
\end{figure}

\section*{Zadanie 25}
Sprawdzenie kodów zakończenia poleceń.
\begin{figure}[H]%
    \centering\includegraphics[width=0.9\textwidth]{251.png}
\end{figure}
\begin{figure}[H]%
    \centering\includegraphics[width=0.9\textwidth]{252.png}
\end{figure}
\begin{figure}[H]%
    \centering\includegraphics[width=0.9\textwidth]{253.png}
\end{figure}
\begin{figure}[H]%
    \centering\includegraphics[width=0.9\textwidth]{254.png}
\end{figure}

\section*{Zadanie 26}
Skrypt wyświetlający listę imion.
\begin{figure}[H]%
    \centering\includegraphics[width=0.9\textwidth]{261.png}
\end{figure}
\begin{figure}[H]%
    \centering\includegraphics[width=0.9\textwidth]{262.png}
\end{figure}

\section*{Zadanie 27}
Skrypt dopisujący 10 razy ten sam tekst.
\begin{figure}[H]%
    \centering\includegraphics[width=0.9\textwidth]{271.png}
\end{figure}
\begin{figure}[H]%
    \centering\includegraphics[width=0.9\textwidth]{272.png}
\end{figure}

\section*{Zadanie 28}
Skrypt pobierający aktualną godzinę co 3 sekundy.
\begin{figure}[H]%
    \centering\includegraphics[width=0.9\textwidth]{281.png}
\end{figure}
\begin{figure}[H]%
    \centering\includegraphics[width=0.9\textwidth]{282.png}
\end{figure}

\section*{Zadanie 29}
Skrypt wyświetlający pliki i ich zawartość.
\begin{figure}[H]%
    \centering\includegraphics[width=0.9\textwidth]{291.png}
\end{figure}
\begin{figure}[H]%
    \centering\includegraphics[width=0.9\textwidth]{292.png}
\end{figure}

\end{document}