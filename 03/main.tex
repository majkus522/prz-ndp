\documentclass[fleqn,onecolumn,a4paper,12pt,titlepage]{article}
\usepackage[showcrop]{geometry}
\geometry{textheight=25.0cm,textwidth=17.0cm,headheight=3cm,headsep=0.5em}%hmargin=4.2cm
\usepackage{graphicx}%[dvips,pdftex]
\usepackage{amssymb}
\usepackage{amsmath}%centertags
\usepackage[T1]{fontenc}
\usepackage[polish]{babel}
\usepackage{float}
\usepackage[final]{listingsutf8}
\usepackage{fancyhdr}
\usepackage{tikz}
\usepackage{caption}
\usepackage{hyperref}
\usepackage{datetime2}
\usepackage{enumitem}
\usepackage{blindtext}
\usepackage{listings}
\usepackage{color}

\definecolor{dkgreen}{rgb}{0,0.6,0}
\definecolor{gray}{rgb}{0.5,0.5,0.5}
\definecolor{mauve}{rgb}{0.58,0,0.82}

\lstset{frame=tb,
  language=Bash,
  aboveskip=3mm,
  belowskip=3mm,
  showstringspaces=false,
  columns=flexible,
  basicstyle={\small\ttfamily},
  numbers=none,
  numberstyle=\tiny\color{gray},
  keywordstyle=\color{blue},
  commentstyle=\color{dkgreen},
  stringstyle=\color{mauve},
  breaklines=true,
  breakatwhitespace=true,
  tabsize=3
}

\pagestyle{fancy}
\fancyfoot[L,R]{\tiny{\texttt{\symbol{64}}  built: \today{} at \DTMcurrenttime}} 

\begin{document}

\begin{titlepage}
    \includegraphics[width=0.25\textwidth]{logo_kia.png}\par\vspace{3cm}
    \centering
    {\LARGE \textsc{Narzędzia dla programistów} \par}
    \vspace{2cm}
    {\Large \textsc{Laboratorium 3} \par} % NALEŻY UZUPEŁNIĆ
    \vspace{2cm}
    {\textsc{Skrypty powłoki} \par} % NALEŻY UZUPEŁNIĆ
    \vfill
    Mikołaj Hus $-$ 179503 $-$ L06 \par % NALEŻY UZUPEŁNIĆ
    \vspace{2cm}
    {\large {\today} \par}
\end{titlepage}

\section*{Zadanie 1}

Uruchomienie programu \textit{Visual Studio Code}
\begin{figure}[H]%
    \centering\includegraphics[width=0.9\textwidth]{01.png}
\end{figure}

\section*{Zadanie 2}

Rozpoczęcie skryptu \textit{Bash}
\begin{figure}[H]%
    \centering\includegraphics[width=0.9\textwidth]{02.png}
\end{figure}

\section*{Zadanie 3}

Utworzenie prostego skryptu wyświetlającego tekst
\begin{figure}[H]%
    \centering\includegraphics[width=0.9\textwidth]{03.png}
\end{figure}

\section*{Zadanie 4}

Próba uruchomienia skryptu
\begin{figure}[H]%
    \centering\includegraphics[width=0.9\textwidth]{04.png}
\end{figure}
Nie została podana ścieżka do pliku przez co Terminal uznał \textit{pkt1} za wbudowaną komendę.

\section*{Zadanie 5}

Ponowna próba uruchomienia skryptu
\begin{figure}[H]%
    \centering\includegraphics[width=0.9\textwidth]{05.png}
\end{figure}
Plik nie posiada atrybutu pozwalającego na jego uruchomienie.

\section*{Zadanie 6}

Zmiana uprawnień pliku.
\begin{figure}[H]%
    \centering\includegraphics[width=0.9\textwidth]{06.png}
\end{figure}

\section*{Zadanie 7}

Poprawne uruchomienie skryptu
\begin{figure}[H]%
    \centering\includegraphics[width=0.9\textwidth]{07.png}
\end{figure}
Skrypt wypisał tekst podany w poleceniu \textit{echo}

\section*{Zadanie 8}

Przygotowanie kolejnego skryptu
\begin{figure}[H]%
    \centering\includegraphics[width=0.9\textwidth]{081.png}
\end{figure}
\begin{figure}[H]%
    \centering\includegraphics[width=0.9\textwidth]{082.png}
\end{figure}

\section*{Zadanie 9}

Utworzenie skryptu sprawdzającego istnienie pliku
\begin{figure}[H]%
    \centering\includegraphics[width=0.9\textwidth]{091.png}
\end{figure}
\begin{figure}[H]%
    \centering\includegraphics[width=0.9\textwidth]{092.png}
\end{figure}

\section*{Zadanie 10}

Sprawdzenie i utworzenie pliku \textit{dane1}
\begin{figure}[H]%
    \centering\includegraphics[width=0.9\textwidth]{101.png}
\end{figure}
\begin{figure}[H]%
    \centering\includegraphics[width=0.9\textwidth]{102.png}
\end{figure}

\section*{Zadanie 11}

Utworzenie pliku sprawdzającego która liczba jest większa
\begin{figure}[H]%
    \centering\includegraphics[width=0.9\textwidth]{111.png}
\end{figure}
\begin{figure}[H]%
    \centering\includegraphics[width=0.9\textwidth]{112.png}
\end{figure}

\section*{Zadanie 12}

Rozszerzenie diałania skryptu o równość liczb
\begin{figure}[H]%
    \centering\includegraphics[width=0.9\textwidth]{121.png}
\end{figure}
\begin{figure}[H]%
    \centering\includegraphics[width=0.9\textwidth]{122.png}
\end{figure}

\section*{Zadanie 13}
\subsection*{A. \normalfont{Sprawdzenie działania skryptu dla niepoprawnych danych}} 
\begin{figure}[H]%
    \centering\includegraphics[width=0.9\textwidth]{131.png}
\end{figure}
Skrypt uznaje nieporawną wartośco jako liczbę 0
\subsection*{B. \normalfont{Zabezpieczenie skryptu przed wprowadzenie niepoprawnych danych}} 
\begin{figure}[H]%
    \centering\includegraphics[width=0.9\textwidth]{132.png}
\end{figure}

\section*{Zadanie 14}

Utworzenie skryptu sprawdzającego zawartość katalogu
\begin{figure}[H]%
    \centering\includegraphics[width=0.9\textwidth]{141.png}
\end{figure}
\begin{figure}[H]%
    \centering\includegraphics[width=0.9\textwidth]{142.png}
\end{figure}
\begin{figure}[H]%
    \centering\includegraphics[width=0.9\textwidth]{143.png}
\end{figure}

\section*{Zadanie 15}

Utworzenie skryptu wypisującego podany tekst n-razy
\begin{figure}[H]%
    \centering\includegraphics[width=0.9\textwidth]{151.png}
\end{figure}
\begin{figure}[H]%
    \centering\includegraphics[width=0.9\textwidth]{152.png}
\end{figure}

\section*{Zadanie 16}

Utworzenie skryptu operującego na datach i logach
\begin{figure}[H]%
    \centering\includegraphics[width=0.9\textwidth]{161.png}
\end{figure}
\begin{figure}[H]%
    \centering\includegraphics[width=0.9\textwidth]{162.png}
\end{figure}

\section*{Zadanie 18}

Otworzenie pliku przy użyciu edytora \textit{VIM}
\begin{figure}[H]%
    \centering\includegraphics[width=0.9\textwidth]{18.png}
\end{figure}

\section*{Zadanie 19}

Testowanie poruszania się po pliku
\begin{figure}[H]%
    \centering\includegraphics[width=0.9\textwidth]{19.png}
\end{figure}

\section*{Zadanie 20}

Powrót do początku pliku
\begin{figure}[H]%
    \centering\includegraphics[width=0.9\textwidth]{20.png}
\end{figure}

\section*{Zadanie 21}

Wyszukiwanie tekstu \textit{echo}
\begin{figure}[H]%
    \centering\includegraphics[width=0.9\textwidth]{21.png}
\end{figure}

\section*{Zadanie 22}

Przejście do ostatniej lini pliku
\begin{figure}[H]%
    \centering\includegraphics[width=0.9\textwidth]{22.png}
\end{figure}

\section*{Zadanie 23}

Usunięcie bierzącej lini
\begin{figure}[H]%
    \centering\includegraphics[width=0.9\textwidth]{23.png}
\end{figure}

\section*{Zadanie 24}

Dodanie nowej lini
\begin{figure}[H]%
    \centering\includegraphics[width=0.9\textwidth]{24.png}
\end{figure}

\section*{Zadanie 25}

Wprowadzenie komentarza
\begin{figure}[H]%
    \centering\includegraphics[width=0.9\textwidth]{25.png}
\end{figure}

\section*{Zadanie 26}

Wstawienie wyciętej lini
\begin{figure}[H]%
    \centering\includegraphics[width=0.9\textwidth]{26.png}
\end{figure}

\section*{Zadanie 27}

Zapisanie pliku
\begin{figure}[H]%
    \centering\includegraphics[width=0.9\textwidth]{27.png}
\end{figure}

\section*{Zadanie 28}

Przejście do 3 lini pliku
\begin{figure}[H]%
    \centering\includegraphics[width=0.9\textwidth]{28.png}
\end{figure}

\section*{Zadanie 29}

Wstawienie podanego tekstu
\begin{figure}[H]%
    \centering\includegraphics[width=0.9\textwidth]{29.png}
\end{figure}

\section*{Zadanie 30}

Opuszczenie trybu wstawiania i zapisanie pliku
\begin{figure}[H]%
    \centering\includegraphics[width=0.9\textwidth]{30.png}
\end{figure}

\section*{Zadanie 31}

Uruchomienie skryptu
\begin{figure}[H]%
    \centering\includegraphics[width=0.9\textwidth]{31.png}
\end{figure}

\section*{Zadanie 32}

Zakończenie pracy edytora
\begin{figure}[H]%
    \centering\includegraphics[width=0.9\textwidth]{32.png}
\end{figure}

\end{document}