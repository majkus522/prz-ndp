\documentclass[fleqn,onecolumn,a4paper,12pt,titlepage]{article}
\usepackage[showcrop]{geometry}
\geometry{textheight=25.0cm,textwidth=17.0cm,headheight=3cm,headsep=0.5em}%hmargin=4.2cm
\usepackage{graphicx}%[dvips,pdftex]
\usepackage{amssymb}
\usepackage{amsmath}%centertags
\usepackage[T1]{fontenc}
\usepackage[polish]{babel}
\usepackage{float}
\usepackage[final]{listingsutf8}
\usepackage{fancyhdr}
\usepackage{tikz}
\usepackage{caption}
\usepackage{hyperref}
\usepackage{datetime2}
\usepackage{enumitem}
\usepackage{blindtext}
\usepackage{listings}
\usepackage{color}

\definecolor{dkgreen}{rgb}{0,0.6,0}
\definecolor{gray}{rgb}{0.5,0.5,0.5}
\definecolor{mauve}{rgb}{0.58,0,0.82}

\lstset{frame=tb,
  language=Bash,
  aboveskip=3mm,
  belowskip=3mm,
  showstringspaces=false,
  columns=flexible,
  basicstyle={\small\ttfamily},
  numbers=none,
  numberstyle=\tiny\color{gray},
  keywordstyle=\color{blue},
  commentstyle=\color{dkgreen},
  stringstyle=\color{mauve},
  breaklines=true,
  breakatwhitespace=true,
  tabsize=3
}

\pagestyle{fancy}
\fancyfoot[L,R]{\tiny{\texttt{\symbol{64}}  built: \today{} at \DTMcurrenttime}} 

\begin{document}

\begin{titlepage}
    \includegraphics[width=0.25\textwidth]{kai.png}\par\vspace{3cm}
    \centering
    {\LARGE \textsc{Narzędzia dla programistów} \par}
    \vspace{2cm}
    {\Large \textsc{Laboratorium 6} \par} % NALEŻY UZUPEŁNIĆ
    \vspace{2cm}
    {\textsc{System kontroli wersji GIT} \par} % NALEŻY UZUPEŁNIĆ
    \vfill
    Mikołaj Hus $-$ 179503 $-$ L06 \par % NALEŻY UZUPEŁNIĆ
    \vspace{2cm}
    {\large {\today} \par}
\end{titlepage}

\section*{Zadanie 1}
Sprawdzenie nazwy użytkownika i adresu email w konfiguracji GIT.
\begin{figure}[H]%
    \centering\includegraphics[width=0.9\textwidth]{01.png}
\end{figure}

\section*{Zadanie 2}
Sprawdzenie numeru stanowiska.
\begin{figure}[H]%
    \centering\includegraphics[width=0.9\textwidth]{02.png}
\end{figure}

\section*{Zadanie 3 - 4}
Ustalenie nazw repozytoriów.
\lstinputlisting{03.txt}

\section*{Zadanie 5 - 7}
Utworzenie folderu roboczego i inicjalizacja repozytorium.
\begin{figure}[H]%
    \centering\includegraphics[width=0.9\textwidth]{05.png}
\end{figure}

\section*{Zadanie 8}
Skopiowanie pliku źródłowego do folderu roboczego.
\begin{figure}[H]%
    \centering\includegraphics[width=0.9\textwidth]{08.png}
\end{figure}

\section*{Zadanie 9}
Dodanie pliku do repozytorium.
\begin{figure}[H]%
    \centering\includegraphics[width=0.9\textwidth]{09.png}
\end{figure}

\section*{Zadanie 10}
Zatwierdzenie wersji.
\begin{figure}[H]%
    \centering\includegraphics[width=0.9\textwidth]{10.png}
\end{figure}

\section*{Zadanie 11}
Sprawdzenie statusu repozytorium.
\begin{figure}[H]%
    \centering\includegraphics[width=0.9\textwidth]{11.png}
\end{figure}

\section*{Zadanie 12}
Wprowadzenie zmian w pliku źródłowym.
\begin{figure}[H]%
    \centering\includegraphics[width=0.9\textwidth]{12.png}
\end{figure}

\section*{Zadanie 13 - 14}
Dodanie pliku i zatwierdzenie zmian.
\begin{figure}[H]%
    \centering\includegraphics[width=0.9\textwidth]{13.png}
\end{figure}

\section*{Zadanie 15}
Sprawdzenie logów repozytorium.
\begin{figure}[H]%
    \centering\includegraphics[width=0.9\textwidth]{15.png}
\end{figure}

\section*{Zadanie 16}
Dodanie etykiety do tej wersji dokumentu.
\begin{figure}[H]%
    \centering\includegraphics[width=0.9\textwidth]{16.png}
\end{figure}

\section*{Zadanie 17}
Dokonanie zmian i ich zatwierdzenie.
\begin{figure}[H]%
    \centering\includegraphics[width=0.9\textwidth]{17.png}
\end{figure}

\section*{Zadanie 18}
Utworzenie kolejnego folderu.
\begin{figure}[H]%
    \centering\includegraphics[width=0.9\textwidth]{18.png}
\end{figure}

\section*{Zadanie 19}
Skopiowanie lokalnego repozytorium.
\begin{figure}[H]%
    \centering\includegraphics[width=0.9\textwidth]{19.png}
\end{figure}

\section*{Zadanie 20}
Umieszczenie pliku źródłowego C w repozytorium.
\begin{figure}[H]%
    \centering\includegraphics[width=0.9\textwidth]{20.png}
\end{figure}

\section*{Zadanie 21}
Dodanie pliku i zatwierdzenie zmian.
\begin{figure}[H]%
    \centering\includegraphics[width=0.9\textwidth]{21.png}
\end{figure}

\section*{Zadanie 22}
Wyświetlenie nazw serwerów.
\begin{figure}[H]%
    \centering\includegraphics[width=0.9\textwidth]{22.png}
\end{figure}

\section*{Zadanie 23}
Dodanie zdalnego repozytorium.
\begin{figure}[H]%
    \centering\includegraphics[width=0.9\textwidth]{23.png}
\end{figure}

\section*{Zadanie 24}
Ustawienie domyślnego celu wysyłania.
\begin{figure}[H]%
    \centering\includegraphics[width=0.9\textwidth]{24.png}
\end{figure}

\section*{Zadanie 25}
Powrót do poprzedniego folderu.
\begin{figure}[H]%
    \centering\includegraphics[width=0.9\textwidth]{25.png}
\end{figure}

\section*{Zadanie 26}
Dodanie zdalnego repozytorium.
\begin{figure}[H]%
    \centering\includegraphics[width=0.9\textwidth]{26.png}
\end{figure}

\section*{Zadanie 27}
Dodanie serwera zdalnego, które zostaje odrzucone przez brak aktualnych zmian. Pobranie i uaktualnienie zmian z serwera.
\begin{figure}[H]%
    \centering\includegraphics[width=0.9\textwidth]{27.png}
\end{figure}

\section*{Zadanie 28}
Modyfikacja pliku \textit{raport.tex}.
\begin{figure}[H]%
    \centering\includegraphics[width=0.9\textwidth]{28.png}
\end{figure}

\section*{Zadanie 29}
Dodanie zmian do repozytorium.
\begin{figure}[H]%
    \centering\includegraphics[width=0.9\textwidth]{29.png}
\end{figure}

\section*{Zadanie 30}
Dokonanie zmian w repozytorium \textit{lok} i próba przesłania zmian na serwer zakończona niepowodzeniem.
\begin{figure}[H]%
    \centering\includegraphics[width=0.9\textwidth]{30.png}
\end{figure}

\section*{Zadanie 31}
Pobranie zmian z serwera oraz rozwiązanie problemów związanych z scalaniem.
\begin{figure}[H]%
    \centering\includegraphics[width=0.9\textwidth]{31.png}
\end{figure}

\section*{Zadanie 32}
Przesłanie scalenia na serwer.
\begin{figure}[H]%
    \centering\includegraphics[width=0.9\textwidth]{32.png}
\end{figure}

\section*{Zadanie 33 - 34}
Utworzenie katalogu roboczego i sklonowanie repozytorium wspólnego.
\begin{figure}[H]%
    \centering\includegraphics[width=0.9\textwidth]{33.png}
\end{figure}

\section*{Zadanie 35}
Wprowadzenie indeksu do pliku \textit{.tex}.
\begin{figure}[H]%
    \centering\includegraphics[width=0.9\textwidth]{35.png}
\end{figure}

\section*{Zadanie 36}
Utworzenie pliku \textit{tresc.txt}.
\begin{figure}[H]%
    \centering\includegraphics[width=0.9\textwidth]{36.png}
\end{figure}

\section*{Zadanie 37}
Zatwierdzenie zmian i przesłanie ich na serwer.
\begin{figure}[H]%
    \centering\includegraphics[width=0.9\textwidth]{37.png}
\end{figure}

\section*{Zadanie 38}
Utworzenie nowej gałęzi repozytorium oraz przesłanie jej na serwer.
\begin{figure}[H]%
    \centering\includegraphics[width=0.9\textwidth]{38.png}
\end{figure}

\section*{Zadanie 39}
Przełączenie gałęzi, pobranie zmian z serwera i scalenie zmian.
\begin{figure}[H]%
    \centering\includegraphics[width=0.9\textwidth]{39.png}
\end{figure}

\section*{Zadanie 40}
Rozwiązanie potencjalnych konfliktów jeżeli takie istnieją.
\begin{figure}[H]%
    \centering\includegraphics[width=0.9\textwidth]{40.png}
\end{figure}

\section*{Zadanie 41}
Sprawdzenie kto zatwierdził daną linię pliku.
\begin{figure}[H]%
    \centering\includegraphics[width=0.9\textwidth]{41.png}
\end{figure}

\section*{Zadanie 42}
Sprawdzenie autorów dla pliku na innej gałęzi.
\begin{figure}[H]%
    \centering\includegraphics[width=0.9\textwidth]{42.png}
\end{figure}

\section*{Zadanie 43}
Scalenie gałęzi pobocznej do gałęzi głównej.
\begin{figure}[H]%
    \centering\includegraphics[width=0.9\textwidth]{43.png}
\end{figure}

\section*{Zadanie 44}
Sprawdzenie gałęzi na serwerze.
\begin{figure}[H]%
    \centering\includegraphics[width=0.9\textwidth]{44.png}
\end{figure}

\section*{Zadanie 42}
Przełączenie do gałęzi głownej i usunięcie gałęzi lokalnej.
\begin{figure}[H]%
    \centering\includegraphics[width=0.9\textwidth]{42.png}
\end{figure}

\end{document}